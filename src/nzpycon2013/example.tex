\documentclass[a4paper, twocolumn]{article}
% generated by Docutils <http://docutils.sourceforge.net/>
\usepackage{fixltx2e} % LaTeX patches, \textsubscript
\usepackage{cmap} % fix search and cut-and-paste in Acrobat
\usepackage{ifthen}
\usepackage[T1]{fontenc}
\usepackage[utf8]{inputenc}
\setcounter{secnumdepth}{0}

%%% Custom LaTeX preamble
% PDF Standard Fonts
\usepackage{mathptmx} % Times
\usepackage[scaled=.90]{helvet}
\usepackage{courier}
\usepackage{minted}

%%% User specified packages and stylesheets

%%% Fallback definitions for Docutils-specific commands

% titlereference role
\providecommand*{\DUroletitlereference}[1]{\textsl{#1}}

% hyperlinks:
\ifthenelse{\isundefined{\hypersetup}}{
  \usepackage[colorlinks=true,linkcolor=blue,urlcolor=blue]{hyperref}
  \urlstyle{same} % normal text font (alternatives: tt, rm, sf)
}{}
\hypersetup{
  pdftitle={Title},
}

%%% Title Data
\title{\phantomsection%
  Title%
  \label{title}}
\author{}
\date{}

%%% Body
\begin{document}

\LARGE
\begin{minted}{rst}
Title
=====

*Some* inline markup 
is :emphasis:`done`
using roles.

.. code::
    
    def plus_one(n):
        return n + 1
        
Subtitle
--------

The default role 
(title-reference):
`Title`. Can be
changed.
\end{minted}

\newpage

\Huge

%\maketitle

\section{\Huge{Title}}

\emph{Some} inline markup is \emph{done} using roles.
%
\begin{quote}{\ttfamily \raggedright \noindent
def~plus\_one(n):\\
~~~~return~n~+~1
}
\end{quote}

\subsection{\huge{Subtitle}}

The default role (title-reference): \DUroletitlereference{Title}. Can be changed.

\end{document}
